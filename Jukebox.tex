%\documentclass[12pt,a4paper]{book}\usepackage{array}\usepackage{tabu}\usepackage{multirow}\newcolumntype{C}[1]{>{\centering\arraybackslash}p{#1}}\usepackage{multirow}\usepackage[utf8]{inputenc}\usepackage[T1]{fontenc}\usepackage{lmodern}\usepackage{hyperref}\usepackage{graphicx}\usepackage[english, frenchb]{babel}\usepackage{listings}	\lstdefinestyle{customstyle}{basicstyle=\footnotesize,breakatwhitespace=false,breaklines=true,captionpos=b,keepspaces=true,tabsize=4,frame=single,moredelim=[is][\underbar]{_}{_}}\lstset{style=customstyle}\usepackage{menukeys}\begin{document}
\chapter{Jukebox}
\section{Do I need that?}
Maybe you want to use your Raspberry as a jukebox controlled remotely, so your 
music will be played on sound devices -- through HDMI or DVI port -- attached to 
your RPi.
\\\\
A MPD\footnote{MPD for Music Player Daemon} server is exactly what you need and in 
addition it can manage playlists -- not explained here -- and it exists a plenty 
of applications\footnote{A list of client are available on mpd wiki, see 
\href{http://mpd.wikia.com/wiki/Clients}{mpd.wikia.com/wiki/Clients}} which are 
dedicated or compatible with. 
\\\\
In following sections you will learn how to setup your server and then you will 
discover four clients for various platforms (PC and smartphones).

\section{MPD setup}\label{section:MpdSetup}

There are two things you need to take care of when you want to install a mpd 
server: how to play music on the RPi and where your music will be stored. 
\\\\
In the example, it is a \og{}classic\fg{} situation wherein you music is located 
in a folder called \emph{music} placed in a user home directory.

\begin{lstlisting}[language=bash,caption=Install mpd server]
# mpd is the server
# alsa-utils manage your RPi audio outputs
#
$ pacman -S mpd alsa-utils

# Update the mpd config file to work with
# the RPi audio (see the next example)
$ nano /etc/mpd.conf

# Mpd files (including music database) are located
# in /var/lib/mpd so you have to change it's owner
# for the mpd user to be able to access it
#
# chown: change file or diretory owner
# "mpd:mpd": new owner (user and his group)
#
$ chown mpd:mpd /var/lib/mpd

# My music is located in the SD card, where the system
# is installed: /home/jeremy/music
#
# In default configuration, your home directory can only 
# be crossed by you but the mpd user must access to your
# music inside
#
# chmod: changes rights on a file or a directory
# o+x: anyone can cross the specified directory
# or execute the file (if it is a file)
#
$ chmod o+x /home/jeremy
\end{lstlisting} 

\lstset{language=bash,caption=Mpd configuration file,label=listing:MpdConfigFile}
\lstinputlisting{code/mpd.conf}

A RPi has two audio outputs (HDMI and DVI) so you can play your music on your 
screen (or TV, or audio system plugged in your TV) and in a headset or any device 
which provides a jack input. 
\\\\
Do not forget the power of a Raspberry, is modularity can help you to setup 
bluetooth or airplay wireless network to play your music. However, these 
interesting solutions will be not covered in this book.
\\
\begin{lstlisting}[language=bash,caption=Set audio output]
# Set audio output to HDMI
# Last number: 0 (auto, avoid it), 1 (DVI), 2 (HDMI)
$ amixer cset numid=3 2

# Fix HDMI troubles
# Uncomment hdmi_force_hotplug=1 and hdmi_drive=2
$ nano /boot/config.txt

# You must reboot for HDMI changes
$ reboot

# Enable, start and then check if the service started
$ systemctl enable mpd
$ systemctl start mpd
$ systemctl status mpd
\end{lstlisting}

\section{Music directory on usb device}
\subsection{Connect an usb device}
Connect a device to an linux system as ArchLinux is called \emph{mount}, once you
connected you device the system will be able to see it with no ways to access it.
\\\\
It is after the \emph{mount} operation that the system get an access to the device 
and programs as mpd will be able to read on.

\begin{lstlisting}[language=bash,caption=Mount an external device]
# List of connected devices
$ fdisk -l
# One device: mmcblk0, zero because you can have others sd cards 
# called mmcblk1, mmcblk2, ...
Disk /dev/mmcblk0 : 7,5 GiB, 8017412096 bytes, 15659008 sectors

# Divided in 2 parts (partitions)
# Note: this table was truncated in lines and columns
Device          Size    Id  Type
/dev/mmcblk0p1  116,2M   e  W95 FAT16 (LBA)
/dev/mmcblk0p2    7,3G  85  Linux extended

# Create a directory in /mnt where you
# will see your device files
$ mkdir /mnt/sdcard

# Mount a device (mmcblk0 for example)
#
# Can add some options:
# -t ntfs-3g: format type of the device
# -o uid=user,gid=group: owner of mounted directory
#
$ mount /dev/mmcblk /mnt/sdcard

# Unmount a device: use the target directory
# choosed on mounting operation
#
# Some devices may be used by programs or services
# Be careful you are not in your device directory (use pwd)
#
$ umount /mnt/sdcard

# List mounted directories
$ mount
/dev/sda1 on /mnt/hd1 type vfat 
\end{lstlisting}

If you use some devices with a Windows system they are maybe formated with the 
\emph{ntfs} format that is not recognize by default on a linux based system.
\\\\
To solve this trouble you have to install \emph{ntfs-3g} which contains 
necessary drivers and an easiest way to mount ntfs devices.
\begin{lstlisting}[language=bash,caption=Mount ntfs devices]
# Install the package with ntfs drivers
$ pacman -S ntfs-3g

# For ntfs-3g works you need to reboot
$ reboot

# Mount your ntfs device
# Short for "mount -t ntfs-3g /dev/mmcblk /mnt/sdcard"
$ ntfs-3g /dev/mmcblk /mnt/sdcard
\end{lstlisting}

\subsection{Change directory in mpd}
To change your music directory in MPD this is only one file to update 
mpd.conf (refer to \ref{listing:MpdConfigFile}) which contains the path to your 
music on \emph{music\_directory} line.
\\\\
Nevertheless, this is one thing you need to take care of: rights on every directory 
in your path. In linux system, there are three access types : read, write and 
execute knowing execute means \og{}can cross\fg{} for directories.

\begin{lstlisting}[language=bash,caption=Analyze rights on home]
$ ls -l / | grep home
drwxr-x--x  3 root root  4096 18 dec.  23:48 home
\end{lstlisting}

Let's analyze the previous result for \texttt{/home} directory to 
able to understand rights easily.

\begin{table}[h]
\centering
	\begin{tabu}{p{2cm}|p{40px}|l|l|p{40px}|l|p{2cm}}
		\rowfont{\bfseries\footnotesize}
		Rights (modes) & Links number & Owner & Group & Size (in bytes) & Last update   & Folder name\\
		&&&&&&\\
		\rowfont{\ttfamily\small}
		drwxr-x--x     & 3            & root  & root  & 4096            & 18 dec. 23:48 & home\\
	\end{tabu}
	\caption{Details on \texttt{ls -l} syntax}
\end{table}

Now, we can decrypt \texttt{ls -l} results and identify who is the owner of a 
file and of which group it belongs. With the first part of the output it is 
possible to deduce what things a user can do on the given file.

\begin{table}[h]
\centering
	\begin{tabular}{rl}
		\hline
		\multicolumn{2}{|c|}{Indicators}\\
		\hline
		d   & File type: directory (d), symbolic link (l), file (-)\\
		&\\
		\hline
		\multicolumn{2}{|c|}{Rights -- nothing(-) read(r) write(w) execute(x)}\\
		\hline
		rwx & Owner, got all rights usually\\ 
		r-w & Users who belongs to group\\ 
		--w & Others\\ 		
	\end{tabular}
	\caption{Rights on a file}
\end{table}

Remember, the initial purpose was to change our mpd music directory so let's 
change the directory to \og{}\texttt{/home/jeremy/music}\fg{} with good rights 
to ensure the \emph{mpd} user will be able to access to the \emph{music} directory 
and read files into.

\begin{lstlisting}[language=bash,caption=Check rights on path]
# We want to check that mpd user can acces
# to /home/jeremy/music, we have to check the three directories
#
$ ls -l / | grep home

# Anyone can cross home: OK
drwxr-xr-x  3 root root  4096 18 dec.  23:48 home

$ ls -l /home | grep jeremy
# Nobody can cross this directory exept jeremy: NOT OK
drwx------ 4 jeremy jeremy 4096 30 dec.  11:56 jeremy

# Add execute right for everyone because mpd do not 
# belongs to jeremy group
#
# Format: level+/-right
#   level: u (owner), g (owner group), o (anyone)
#   +/-: + to add right, - to remove it
#   right: r (read), w (write), x (execute)
#
$ chmod o+x /home/jeremy

$ ls -l /home | grep jeremy
# Anyone can cross this directory: OK
drwx-----x 4 jeremy jeremy 4096 30 dec.  11:56 jeremy

$ ls -l /home/jeremy/ | grep music
# Anyone can read music files into this directory: OK
drwxr-xr-x 2 jeremy jeremy 4096 29 dec.  12:18 music

# Now the full path has been verified: mpd can cross each
# directories from root and read in music folder
\end{lstlisting}

\section{Classic and web clients}

There many mpd clients available and it is necesarry to distinguish thoses which 
are installed on the server -- as web clients -- and others that are on a remote 
machine owned by the client.

\subsection{External clients}
I will present you two applications which are mpd client and as much as possible 
multiplatform or at least compatible with most common devices.
\\\\
The first one is \texttt{MPDroid} an Android application available on the play 
store\footnote{Download MPDroid on the play store at \href{http://play.google.com/
store/apps/details?id=com.namelessdev.mpdroid}{play.google.com/store/apps/details?
id=com.namelessdev.mpdroid}} which enable you to manage your mpd server depending 
on the wireless network you are connected.
\\\\
GMPC\footnote{Gnome music player client} is the second client that I will talk 
about because it's strong point is that it is multiplateform so compatible with 
Windows, Mac OSX and Linux systems.
\\\\
To setup the connexion between your application and the mpd server on the RPi the 
only thing you need is it's IP address or domain name. That is why we did not 
change the default port -- 6600 -- and set a password to access it but feel free 
to do it if it is necessary for you.

\subsection{YMPD web client}
YMPD is a lightweight and easy to install web client interface build over the 
bootstrap. It means the interface is really responsive and it can be used 
either on the PC or on a smartphone through a web browser.

\begin{lstlisting}[language=bash,caption=ympd setup]
# wget: tool to download files from network
$ pacman -S wget

# Use wget to download ympd release
$ wget http://www.ympd.org/downloads/ympd-1.2.3-armhf.tar.bz2

# Unpack ympd from the downloaded archive
$ tar -xvf ympd-1.2.3-armhf.tar.bz2

$ rm ympd-1.2.3-armhf.tar.bz2 # Remove unused archive
$ mv ympd /usr/bin            # Move executable to the "classic" directory

# Run ympd server and choose the port you will
# use to connect to it
$ ympd --webport 2020 # Any port > 1024
\end{lstlisting}

Now, you can connect to the ympd web interface from any web browser access to 
\href{http://RPiIP:2020}{http://RPiIP:2020} but pay attention to use the port 
you choosed when you launched ympd your server.

\subsection{Ampache}
If you are looking for a more complex and customizable solution for your mpd 
server web client, Ampache can be the answer to your problem.
\\\\
It is php based application (similar to phpMyAdmin) which allows you to manage 
music catalogs  -- on your server and on remote machines -- and play them on the 
server or locally in your browser (with streaming).

\begin{lstlisting}[language=bash,caption=Download Ampache]
# You will need unzip to unpack ampache 
$ pacman -S unzip

# Download last ampache release
$ wget https://github.com/ampache/ampache/archive/develop.zip

# Unpack ampache, remove zip file and rename ampache directory
$ unzip develop.zip 
$ rm develop.zip
$ mv ampache-develop ampache

# Create a domain with nginx by using the following
# config file (too long but necessary) available at :
# http://github.com/ampache/ampache/wiki/Installation#nginx
#
# See Subdomains and site setup in Web server chapter of this book
# if you have some troubles
\end{lstlisting}

Ampache offers a php script called \emph{install.php} that you can reach by going 
to \href{http://RPiIP/install.php}{http://RPiIP/install.php} but in my case it 
allways redirect me to \emph{/test.php} which tells me that I do not respect 
prerequires.
\\\\
Finally, once I launched install script after some setup, I have realized that 
pretty much all things I have done before had to be done by this script. 
\\\\
As a result either you are lucky and the install script works or follow me for 
some setup to fill requirements and tell to the install script to do nothing 
because we have done it's work.

\begin{lstlisting}[language=bash,caption=Ampache setup]
# The test page show me 6 errors at the begin
# here are the "name" of theses tests plus the 
# ways to pass them

# MySQL and PHP iconv extension
# Problem: php extensions not enabled
#
# Remove ";" before: extension=pdo_mysql.so
#                     extension=iconv.so
#
$ nano /etc/php/php.ini
$ systemctl reload php-fpm

# Configuration file readibility and validity
# Problem: wrong name for ampache config file
#
$ cd /usr/share/webapps/ampache/config
$ mv ampache.cfg.php.dist ampache.cfg.php

# Database connexion
# Problem: user account for database not set in 
# ampache configuration file
#
# Fill following fields with right values
# (remember you may changed the root password)
#        database_username = root
#        database_password = password
#
$ nano /usr/share/webapps/ampache/config/ampache.cfg.php


# Database table
# Problem: ampache database and tables not created
#
$ mysql -u root -ppassword # no space between -p and password
> create database ampache;
> use ampache;
> source /usr/share/webapps/ampache/sql/ampache.sql;
> exit
\end{lstlisting}

At this point, Ampache let you run the install script but it says some tests about 
the configuration files fails, ignore them and then skip database and configuration 
file steps. The last thing to do is creating an admin account to connect yourself 
to ampache web client.

\section{Bluetooth notes}
\begin{lstlisting}[language=bash,caption=Bluetooth]
$ pacman -S bluez bluez-utils
\end{lstlisting}
%\end{document}
