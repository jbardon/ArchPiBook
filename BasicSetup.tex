\documentclass[12pt,a4paper]{book}\usepackage[utf8]{inputenc}\usepackage[T1]{fontenc}\usepackage{lmodern}\usepackage{hyperref}\usepackage{graphicx}\usepackage[english, frenchb]{babel}\usepackage{listings}	\lstdefinestyle{customstyle}{basicstyle=\footnotesize,breakatwhitespace=false,breaklines=true,captionpos=b,keepspaces=true,tabsize=4,frame=single}\lstset{style=customstyle}\begin{document}
\chapter{Basic setup}

The default username and password for ArchLinux is \texttt{root/root}, the
root user got all right on the system it means he can do anything -- even
break the system -- so it is not recommanded to use it.
\\
\begin{lstlisting}[language=bash,caption=Create a new user called jeremy]
$ useradd -m jeremy # Create user jeremy and his home folder
$ passwd jeremy     # Modify jeremy password

# Attribute rights with specific format
$ visudo
# Add new line: "jeremy ALL=(ALL) ALL"
\end{lstlisting}

Now we have a new user account for everydays usage, it is possible to specify
his rights with \texttt{visudo} command. 
The general format is the following \og{}\texttt{username machine=(targetuser) 
commands}\fg{}
\\let me give you some details:

\begin{description}
\item[username] name you gave to \texttt{useradd} command
\item[machine] on which machine rights are applied, ALL in general
\item[target user] user that we take the rights
\item[command] allowed commands separated with one coma -- no spaces --, use exclamation mark for banned commands
\end{description}
\end{document}
