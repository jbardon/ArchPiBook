\documentclass[12pt,a4paper]{book}\usepackage{array}\usepackage{tabu}\usepackage{multirow}\newcolumntype{C}[1]{>{\centering\arraybackslash}p{#1}}\usepackage{multirow}\usepackage[utf8]{inputenc}\usepackage[T1]{fontenc}\usepackage{lmodern}\usepackage{hyperref}\usepackage{graphicx}\usepackage[english, frenchb]{babel}\usepackage{listings}	\lstdefinestyle{customstyle}{basicstyle=\footnotesize,breakatwhitespace=false,breaklines=true,captionpos=b,keepspaces=true,tabsize=4,frame=single,moredelim=[is][\underbar]{_}{_}}\lstset{style=customstyle}\usepackage{menukeys}\usepackage{comment}\begin{document}
\chapter{Git server}
\section{Why would you use Git}
For some reasons, you need to keep several versions of documents so you make 
copies -- can be long and space consumer -- but it still difficult to get the 
last version when more than one person works on the same document.
\\\\
Git is a versionning tool which can do this work for you and in addition keep 
traces of who modified what on each document in a project. This tool will be not  
intrusive and can be used at every moments of a projet -- even if it is allready 
started -- it ensure that you have a project which is the \og{}good version\fg{} 
everytime and it is up-to-date.
\\\\
Largely used in programming projects it can be usefull for any types of projects 
which contains files readable by a simple text editor. You can get an idea of what 
is git by browsing projects on \href{http://github.com}{github.com} which is the 
most famous git projects host because it is free for public (open source) projects.

\section{Install Git and add projects}
To visualize git projects on web interface the first thing you need to have is a 
git project. In order to create our own projet we will download an existing one 
to have a real history and some things to visualize. 
\\\\
I choose to use an open source project -- Firefox OS\footnote{Firefox OS is the 
operating system developed by the Mozilla foundation for smartphones. More 
informations on \href{https://www.mozilla.org/en-US/firefox/os/}{www.mozilla.org/
en-US/firefox/os/}, sources availables on \href{https://github.com/mozilla/
firefox-ios}{github.com/mozilla/firefox-ios}} -- because it is quite fast to 
download and have some folders with enough commits to be interesting.

\begin{lstlisting}[language=bash,caption=Install git and add projects]
# Install git package
$ pacman -S git

# Create repositories(projects) directory
$ mkdir /srv/git

# Get an existing project on github to test our web 
# interface, Firefox OS in this example.
$ cd /srv/git
$ git clone https://github.com/mozilla/firefox-ios.git
\end{lstlisting}

There are a few directories that are most used for repositories but I retained two 
of them: \texttt{/home/git/} and \texttt{/srv/git/}. 
\\\\
In ArchLinux, websites and ftp are by default in \emph{/srv}, and \emph{/home/git} 
is the home directory for \emph{git} user -- which is not a physic user -- so to 
keep a kind of consistency -- in my opinion -- I prefer the second alternative. 

\section{Host repositories}
\subsection{Create a new project}
Host a git project is quite easy because it is really close from creating a local 
git repository. You just must think about adding \emph{--bare} option to specify 
that your project is not a working copy. 
\begin{lstlisting}[language=bash,caption=Create a repository on your server]
# Let's create a new git project called "project"
# Naming convention: add .git at the end of your project name
# to specify it is not a working copy
#
$ mkdir /srv/git/project.git
$ cd /srv/git/project.git
$ git init --bare # --bare: not a working copy
\end{lstlisting}

On a remote machine on which you want to work, clone the project and do a first 
commit on your project.
\begin{lstlisting}[language=bash,caption=Clone repository to work]
# Clone project
# username is your username for local account on the server 
$ git clone username@RPiIP:/srv/git/project.git

# Create a new file and commit
$ cd project
# New file called README.md with "New project" text inside
$ echo "New project" > README.md 
$ git commit -am"First commit"
$ git push origin master # First time, after just "git push"
\end{lstlisting}

\subsection{Use an existing project}
If you want to host a git project which allready exists it is possible to do. 
There two possible situations:
\begin{itemize}
	\item The project is allready hosted by someone else or websites like GitHub 
		  or BitBucket.
	\item It is a local project so hosted locally
\end{itemize}

\subsubsection{Allready hosted}
As you can imagine, the first situation is the easier to solve, you just have to 
clone the repository from the server with a specific option.

\begin{lstlisting}[language=bash,caption=]
$ git clone username@RPiIP:/srv/git/project.git --bare
\end{lstlisting}

Note that the folder you will get will be named \emph{project.git} so it is not a 
working copy. In addition, notice that is the same \emph{--bare} option as when 
you create a new repository on the server.

\subsubsection{Local project}
The first thing to do is retrieve the local project on the server. There are many 
ways to it, I advice to use FTP -- if you setup an FTP -- but for others here is 
a simple way if you have a linux system.

\begin{lstlisting}[language=bash,caption=]
# Your project is named project
$ scp username@RPiIP:/srv/git /local/path/to/project
\end{lstlisting}

With the working copy on your server, the last thing to do is clone the repository 
as \emph{bare} locally and remove the working copy.

\begin{lstlisting}[language=bash,caption=Host repository from local copy]
# Extract project informations
$ git clone --bare /srv/git/projet /srv/git/project.git
$ rm -r /srv/git/project
\end{lstlisting}
% Change remote ?
	
% Different from clone apparently
\begin{comment}
	\begin{lstlisting}[language=bash,caption=Host repository from local copy]
# Extract project informations
$ mv /srv/git/project/.git /srv/git/project.git
$ rm -r /srv/git/project

# Declare repository as bare
$ cd /sr	v/git/project.git
$ git config --bool core.bare true
	\end{lstlisting}
\end{comment}
\section{Web application for viewing}
\subsection{GitWeb: Build-in solution}
Gitweb is the web interface offered by git when you install it. Based on cgi 
technology, it is a bit slow but very refined so easy to use and understand.
\\\\
You can see a live usage of git web at \href{http://repo.or.cz/w?a=project\_list}
{repo.or.cz/w?a=project\_list}, browse projets, commit and all things you want and 
before abandoning GitWeb keep in mind that we will improve a little bit the visual 
aspect.

\subsubsection{Installation and configuration}
\begin{lstlisting}[language=bash,caption=Gitweb setup]
# Gitweb is a cgi script, you need to install 
# some packages
$ pacman -S fcgiwrap spawn-fcgi

# Update php configuration to accept .cgi files
$ nano /etc/php/php-fpm.conf
# Uncomment and add .cgi
security.limit_extensions = .php .php3 .php4 .php5 .cgi

# Set your repositories folder in gitweb.conf
# Look at the next example
$ nano /etc/gitweb.conf  # Is created of not exist
\end{lstlisting}

\lstset{language=bash,caption=gitweb.conf file}
\lstinputlisting{code/gitweb.conf}

\subsubsection{Configure nginx and install cgi tools}
According to the ArchLinux wiki, it is necessary to change the \emph{fcgiwrap.service} 
file because it uses spawn-fcgi so trust the wiki. Note you have to restart 
fcgiwrap service after your updates.

\lstset{language=bash,caption=fcgiwrap.service from wiki.archlinux.org/index.php/gitweb}
\lstinputlisting{code/fcgiwrap.service}

The standard subdomain file presented before will not work in this case because 
gitweb does not contains php scripts but only one cgi script.

\lstset{language=bash,caption=Gitweb configuration for nginx}
\lstinputlisting{code/gitweb}

\subsubsection{Finishing touch}
After that, you will be able to access to your gitweb interface with the domain 
you defined -- \texttt{gitweb.local.pi} in the example -- but if do not see any 
projects although you set some of them in the directory you specified in 
\emph{gitweb.conf}, try to launch git daemon.

\begin{lstlisting}[language=bash,caption=]
$ /usr/lib/git-core/git-daemon --inetd --export-all --base-path=/srv/git
\end{lstlisting}

\mbox{}\\\\
The last thing you can do is replace the default theme the one developed by 
kogakure\footnote{More informations on his web site: \href{http://kogakure.github
.io/gitweb-theme/}{kogakure.github.io/gitweb-theme/}} which is a little bit more 
appealing.

\begin{lstlisting}[language=bash,caption=Install kogakure gitweb theme]
# Save default theme
$ cp -r /usr/share/gitweb/static/ /usr/share/gitweb/static.old

# Retrieve kogakure theme, copy it in right folder
# and delete unused files
$ git clone https://github.com/kogakure/gitweb-theme.git
$ cp gitweb-theme/git* /usr/share/gitweb/static
$ rm -r gitweb-theme
\end{lstlisting}

\subsection{GitList: Simple GitHub clone}

If you are looking for a git web interface which look like GitHub -- with less 
side features -- GitList is probably a good alternative for you. 
\\\\
This project is hosted by GitHub at \href{https://github.com/klaussilveira/gitlist}
{github.com/klaussilveira/gitlist} but as said in the README file, do not clone 
the repository and prefer download the packaged version on the GitList website at 
\href{http://gitlist.org}{gitlist.org}.

\begin{lstlisting}[language=bash,caption=Download GitList]
# Download
$ cd /usr/share/webapps
$ wget https://s3.amazonaws.com/gitlist/gitlist-0.5.0.tar.gz

# Extract and remove unused files
$ tar -xvf gitlist-0.5.0.tar.gz
$ rm gitlist-0.5.0.tar.gz
\end{lstlisting}

Let's declare a new subdomain for GitList, a sample is available in it's directory 
-- \texttt{/usr/share/webapps/gitlist/INSTALL.md} -- but here I will propose you 
a shorter solution. Note you have to use at least one of these two sample because 
GitList uses url rewriting.

\lstset{language=bash,caption=GitList configuration for nginx}
\lstinputlisting{code/gitlist}

Now you can access to GitList interface but you may have some errors messages and 
your are forced to fix these troubles.

\begin{lstlisting}[language=bash,caption=Fix GitList troubles]
# Message 1
# The "/usr/share/webapps/gitlist/cache" folder must be writable 
# for GitList to run.
#
$ mkdir /usr/share/webapps/gitlist/cache # If not exist
$ chown http:http /usr/share/webapps/gitlist/cache

# Message 2
# Please, create the config.ini file.
$ cd /usr/share/webapps/gitlist/
$ mv config.ini-example config.ini

# Message 3
# Please, edit the config file and provide your repositories directory
# Solution: set repositories[] = '/srv/git/'
#
$ nano /usr/share/webapps/gitlist/config.ini
\end{lstlisting}

\section{Gitolite: manage projects}

In the case wherein you have few repositories and you do not need to 
setup access constraints the git common usage will be sufficient.
\\\\
However, if you are looking for a tool which enables you to restrict 
access to repositories for specific persons -- or even groups --  
gitolite is the answer to your problem.
\\\\
The documentation for gitolite is available at \href{http://gitolite.com}
{gitolite.com} and it is well written and understandable. During 
your research may be you found a similar tool called \emph{gitosis} 
which is the ancestor of gitolite. It appears that gitosis is no longer 
used and developped so that why I choose to install gitolite.

\subsection{Installation}

Gitolite will use \emph{git} user and functionalities so it is 
to install git first. In this configuration I made the choice to 
store git repositories -- and gitolite configuration -- in 
\texttt{/srv/git} which will be the home directory of \emph{git} 
user.

\begin{lstlisting}[language=bash,caption=Gitolite installation]
# Download and install gitolite
$ cd /tmp
$ git clone git://github.com/sitaramc/gitolite
$ gitolite/install -to /bin
$ rm -r gitolite

# Create git user directory
$ mkdir /srv/git
$ chown git:git /srv/git

# Set /srv/git as git use home
# Add path, the line for git user should looks like :
#     git:x:997:997:git daemon user:/srv/git:/bin/bash
$ nano /etc/passwd
\end{lstlisting}

A this point, gitolite has been installed but it not yet usuable. 
The last thing to do is defining the administrator which will be able 
to manage repositories access through the \emph{gitolite-admin} 
repository.
\\\\
Interactions with the git server will be made through ssh with the 
git user. As a result, the only way to authenticate users is with 
their ssh public key so we first need to get that key in order 
to define the administator for gitolite.

\begin{lstlisting}[language=bash,caption=Get ssh public key and define gitolite administrator]
# On administrator machine

# Generate key, press enter for each questions
$ ssh-keygen -t rsa

# Copy ssh public key to the server as admin.pub
$ scp ~/.ssh/id_rsa.pub jeremy@local.pi:/srv/git/admin.pub

# On the server as git user
# Connect and configure administrator
$ ssh git@RpiIP
$ cd /srv/git
$ gitolite setup -pk admin.pub
\end{lstlisting}


%https://github.com/sitaramc/gitolite
\begin{lstlisting}[language=bash,caption=Get ssh]
# Client admin
$ git clone ssh://git@local.pi/gitolite-admin
$ nano gitolite-admin/conf/gitolite.conf

repo gitolite-admin
    RW+     =   admin

repo testing
    R       =   admin # try remove W

\end{lstlisting}

% # Uninstall
% http://gitolite.googlecode.com/git-history/f023591183365980d11c1ba352461bea9863746f/doc/uninstall.html
% http://gitolite.com/gitolite/emergencies.html

\subsection{Setup repositories}
\subsection{Adapt GitWeb and GitList}


\end{document}
