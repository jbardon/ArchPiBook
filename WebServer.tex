%\documentclass[12pt,a4paper]{book}\usepackage{array}\usepackage{multirow}\newcolumntype{C}[1]{>{\centering\arraybackslash}p{#1}}\usepackage{multirow}\usepackage[utf8]{inputenc}\usepackage[T1]{fontenc}\usepackage{lmodern}\usepackage{hyperref}\usepackage{graphicx}\usepackage[english, frenchb]{babel}\usepackage{listings}	\lstdefinestyle{customstyle}{basicstyle=\footnotesize,breakatwhitespace=false,breaklines=true,captionpos=b,keepspaces=true,tabsize=4,frame=single,moredelim=[is][\underbar]{_}{_}}\lstset{style=customstyle}\usepackage{menukeys}\begin{document}
\chapter{Web server}
\section{Nginx versus Apache}
In web servers world there are three major actors: Apache, nginx and 
IIS (by Microsoft). If you look at the number of production uses\footnote{Netcraft 
survey available on \href{http://news.netcraft.com/archives/2014/02/03/february-
2014-web-server-survey.html}{news.netcraft.com/archives/2014/02/03/february-2014-
web-server-survey.html}} you will see that I followed the distribution from above 
it means \textasciitilde{}55\% for our old Apache and \textasciitilde{}15\% for 
the two others.
\\\\
However, it appears like Apache decreases by 5\% and nginx get these same 5\% all 
that between January 2014 and February 2014. Now, I can tell you I choosed to use 
nginx instead of Apache for some reasons and the nginx rise is one of them.
\\\\
Let's sum up some advantages and drawbacks of using nginx instead of apache. Few 
of them are extracted from raspbian-france.fr\footnote{Article on the same topic 
(in french), see \href{http://raspbian-france.fr/installer-nginx-raspbian-
accelerez-serveur-web-raspberry}{raspbian-france.fr/installer-nginx-raspbian-
accelerez-serveur-web-raspberry}} but others are personnal. 
\\\\
\begin{table}
\label{table:AvantagesNginx}
\centering
	\begin{tabular}{C{1cm}|p{9cm}}
		\multirow{4}{*}{+} 
		& Asynchronous server\footnote{Parallel execution of tasks}\\
		& More scalable\footnote{Ability to support a lot of connexions in one time}\\
		& Less RAM usage (think about our poor RPi)\\
		& Config file syntax (XML vs readeable)\\
		\hline
		\multirow{4}{*}{-} 
		& Less mature (1995 vs 2004)\\ 
		& Some PHP behaviours may differ from Apache\\
		& .htacess not work directly with nginx\\
	\end{tabular}
	\caption{Advantages and drawbacks of nginx}
\end{table}
\newpage
\section{Installation}
To run a non-static website, nginx is not sufficient we also need to install PHP 
-- for dynamic pages -- and a mySQL database to save our precious data. In 
addition, we will install phpMyAdmin which is a tool -- written in PHP -- to 
manage a mySQL database through a web interface. 

\begin{lstlisting}[language=bash,caption=Install a full web server]
# php-fpm is the nginx module for PHP
# mariadb is a mySQL database
$ pacman -S nginx php-fpm mariadb phpmyadmin

# Enable and run services
$ systemctl enable nginx php-fpm mysqld
$ systemctl start nginx php-fpm mysqld
\end{lstlisting} 

Nginx server works well but php is not enabled in configuration files. Moreover, 
there are some incompatibilties troubles in initial setup.

\begin{lstlisting}[language=bash,caption=Enable php in nginx]
# First problem: welcome file directory is not set in
# availables directories for php
#
# Update php config file
# Search open_basedir, add ":/usr/share/nginx/html/" at the end
#
$ nano /etc/php/php.ini

# Second problem: php not enabled and contains bad setup
# in nginx configuration file
#
# Uncomment (remove #) on the following block and change it 
# to fit the example
#
$ nano /etc/nginx/nginx.conf

# pass the PHP scripts to FastCGI server listening on 127.0.0.1:9000
#
location ~ \.php$ {
    root   /usr/share/nginx/html; # Fix path to php files

# Use sockets   
#   fastcgi_pass   127.0.0.1:9000; 
    fastcgi_pass   unix:/var/run/php-fpm/php-fpm.sock;
            
    fastcgi_index  index.php;

# Use fastcgi.conf file
#   fastcgi_param  SCRIPT_FILENAME  /scripts$fastcgi_script_name;
#   include        fastcgi_params;
    include        fastcgi.conf;
}

# Third problem: index pages must be .html or .html
# want to add index.php
#
location / {
    root           /usr/share/nginx/html;
	
	# Add index.php
    index          index.html index.htm index.php;
}
\end{lstlisting}

Php does not enable mysql extensions by default in configuration files. Therefore, 
to be able to perform SQL queries on local database through php scripts you need 
to enable these extensions.
\begin{lstlisting}[language=bash,caption=Enable php in nginx]
$ nano /etc/php/php.ini

# Remove ";" before mysql and mysqli extensions
;extension=mysql.so
;extension=mysqli.so
\end{lstlisting}

The last thing to do for security purposes is to change mySQL default root 
password that is empty.
\begin{lstlisting}[language=bash,caption=Change mySQL root password]
# -u: considerer specified user
# password: update user password
# NewPassword: your password for root user
#
$ mysqladmin -u root password NewPassword
\end{lstlisting}

Each time you change nginx or php configuration files you have to reload and 
restart the corresponding service for it to take into account your changes.
\begin{lstlisting}[language=bash,caption=Reload and restart a service]
# Restart only considered service
$ systemctl reload nginx php-fpm 
$ systemctl restart nginx php-fpm
\end{lstlisting}

\section{Basic verifications}
The next step is to check our web server with basic verifications to test whether
the installation worked well. We will simply try to access to the default nginx 
page remotely, create a php page and display it and finally access to local 
database. 

\subsection{Nginx and php}
During installation process, nginx places a welcome page called \emph{index.html} 
in \texttt{/usr/share/nginx/html} directory. However Php gives no sample to test 
with nginx therefore we will create a new php script in welcome page directory.
\\\\
The content of this new script will simply display php configuration on 
your server.
\lstset{language=php,caption=Php simple script}
\lstinputlisting{code/index.php}

The goal of these test is to access to \emph{index.html} and \emph{index.php} 
from a remote web browser to check if nginx and php work well on your server. 
\\\\
To perform that, type the IP address of your RPi on your PC web browser -- 
followed by /index.html and then /index.php -- and you will see nginx welcome page 
and php config respectively.

\subsection{MySQL}
There are two points to check with mySQL installation, first the database itself 
and then communication between mySQL and PHP.
\begin{lstlisting}[language=bash,caption=Test mySQL installation]
$ mysql -uroot    # Connect to local database with root user
> show databases; # List of databases
> use mysql;      # Go into mysql database
> show tables;    # List of tables in current database

# Perform SQL query to list users
> select host, user, pasword from user;
> exit; # Logout from local database

# If nothing works try to launch mysql install script
# user: user account used by mysql service
# ldata: path to MariaDB data directory
# basedir: path to MariaDB installation directory
$ mysql_install_db --user=mysql --ldata=/var/lib/mysql 
    --basedir=/usr
\end{lstlisting}

Yet you suceed to list database users with the dedicated command, the last step 
is to check if we can do the same thing through php.
\lstset{language=php,caption=Test php with mySQL}
\lstinputlisting{code/checkdb.php}

\section{Domains and custom directory}
\subsection{Custom web directory}
Nginx default setup set this directory as the root of 
your server but websites are usually hosted in \texttt{/var/www} so I will explain 
you how to move the welcome page into that new file.

\begin{lstlisting}[language=bash,caption=Move nginx welcome page]
# Create /var/www directory and move into it
$ cd /var
$ mkdir www # Create a folder named "www"
$ cd mkdir

# Copy directory (with -r) source destination
$ cp -r /usr/share/nginx/html/ /var/www/

# Move a directory but if source and destination
# are in same directory rename source
$ mv /var/www/html /var/www/home
\end{lstlisting} 

Before trying to access to your website it will be necessary to configure nginx 
and php for the new path. The sample nginx configuration file contains some 
help to do specific things so we will save it and create a new one with the 
following example.

\begin{lstlisting}[language=bash,caption=Configure nginx and php]
# Save and change nginx file
$ mv /etc/nginx/nginx.conf /etc/nginx/nginx.conf.old
$ nano /etc/nginx/nginx.conf # See following example

# Update php config file
# Search open_basedir and add ":/var/www/" at the end
$ nano /etc/php/php.ini

# Do not forget to reload and restart services
\end{lstlisting} 
\lstset{language=bash,caption=nginx configuration}
\lstinputlisting{code/nginx.conf}

\subsection{Subdomains and site setup}
Nginx uses the same principles as Apache in sites management. Indeed, in nginx 
configuration directory -- \texttt{/etc/nginx/} -- there are two folders called 
respectively \emph{site-available} and \emph{site-enabled} and in 
\emph{nginx.conf} file there must be a line which loads sites in 
\emph{site-enabled}.

\begin{lstlisting}[language=bash,caption=Nginx domains architecture]
# First check: site-available and site-enabled exist
#
# ls (list segment): list of files in given directory 
# "|": pass result to the next command
# grep site: display lines which contains "site"
# 
$ ls /etc/nginx | grep site

# Create these folders if there are not displayed
#
# mkdir: create directory with specified name
#
$ mkdir /etc/nginx/site-available /etc/nginx/site-enabled

# Second check: site-enable is included in nginx.conf
#
# cat: display specified file in console
#
$ cat /etc/nginx/nginx.conf | grep site

# If not in nginx.conf add the following line:
#   include /etc/nginx/site-enabled/*;
# in html block
#
$ nano /etc/nginx/nginx.conf
\end{lstlisting} 

Once the nginx setup completed, you can follow with the creation of a subdomain. 
As an example, we would like to access to phyMyAdmin interface -- see 
\ref{section:PhpMyAdmin} for installation -- with \href{http://pma.local.pi}
{\texttt{http://pma.local.pi}} instead of using \href{http://RpiIP/phpmyadmin/}
{\texttt{http://RpiIP/phpmyadmin/}}. 

\begin{lstlisting}[language=bash,caption=Create a subdomain]
# Create phpmyadmin file with the next example as content
$ nano /etc/nginx/site-available/phpmyadmin

# Create a link in site-enable to declare it as enabled
$ ln -s /etc/nginx/site-available/phpmyadmin /etc/nginx/site-enabled/

# Check the link
$ ls -l /etc/nginx/site-enabled/

# Do not forget to restart service
\end{lstlisting} 

\lstset{language=bash,caption=PhpMyAdmin domain example}
\lstinputlisting{code/phpmyadmin}

The final step to be able to access to \href{http://pma.local.pi}
{\texttt{http://pma.local.pi}} is to match this domain with the RPi IP address. 
Usually, there two ways to perform that kind of things :
\begin{description}
	\item[DNS\footnotemark server]\footnotetext{DNS for Domain Name Service} 
					  A service with matches 
					  domain names with IP address. If your router is the server 
					  it will work for all connected devices.
					  
	\item[hosts file] Local file in your machine which overrides DNS server and 
					 thus works only for your machine.\\
\end{description}

For the first method it depends on your material but for the second I can tell you 
that on Linux and Mac OSX \emph{hosts} file is located in \texttt{/etc/hosts} and 
on Windows in \texttt{C:\textbackslash{}Windows\textbackslash{}system32
\textbackslash{}drivers\textbackslash{}etc\textbackslash{}hosts}. 
\\\\
Add a line like \og{}\texttt{RpiIP pma.local.pi}\fg{} in this file opened with 
root access (for Windows launch text editor with administrator rights).

\section{PhpMyAdmin setup}\label{section:PhpMyAdmin}
The installation package puts phpMyAdmin files in \texttt{/usr/share/webapps/
phpMyAdmin} directory so if you want to access to the web interface you need to 
move this directory in \texttt{/usr/share/nginx/html}. However, it would be more 
logical to let application files where there are, so we will create a link.

\begin{lstlisting}[language=bash,caption=Configure phpMyAdmin]
# ln creates a link to source in destination
# -s symbolic link (not change link if source has been moved)
#
$ ln -s /usr/share/webapps/phpMyAdmin/ /usr/share/nginx/html/phpmyadmin
\end{lstlisting} 

Now, you can try to connect to \texttt{RPiIP/phpmyadmin} from a remote 
browser and use the root account of mySQL (remember you changed the default 
password).
\\\\
If you get a warning message which tells you it can not access 
\emph{config.inc.php} it is because phpMyAdmin puts this file in it's 
installation directory and also in \texttt{/etc/webapps/phpmyadmin} but it has 
no rights on the second one.

\begin{lstlisting}[language=bash,caption=Fix phpMyAdmin troubles]
# Update php config file
# Search open_basedir and add ":/etc/webapps/" at the end
$ nano /etc/php/php.ini

# Do not forget to reload and restart php service
\end{lstlisting} 
%\end{document}
