\documentclass[12pt,a4paper]{book}\usepackage{array}\usepackage{tabu}\usepackage{multirow}\newcolumntype{C}[1]{>{\centering\arraybackslash}p{#1}}\usepackage{multirow}\usepackage[utf8]{inputenc}\usepackage[T1]{fontenc}\usepackage{lmodern}\usepackage{hyperref}\usepackage{graphicx}\usepackage[english, frenchb]{babel}\usepackage{listings}	\lstdefinestyle{customstyle}{basicstyle=\footnotesize,breakatwhitespace=false,breaklines=true,captionpos=b,keepspaces=true,tabsize=4,frame=single,moredelim=[is][\underbar]{_}{_}}\lstset{style=customstyle}\usepackage{menukeys}\usepackage{comment}\begin{document}
\chapter{Download box}
Many people uses their servers as seed boxes -- for torrent -- or an intermediate 
to download their files faster because a server has an high speed connexion. 
Considering you RPi will be behind you internet connexion it will be totally useless 
for us.
\\\\
However it allows you to download big files -- for torrent with poor sources -- 
without leaving your PC switched on all night and the RPi is a very low energy 
consumer. In addition combined with an FTP or a file sharing system all your 
downloads will be accessible for other people on your network.

\section{Direct download with aria2}
aria2 is download manager which is able to handle http an torrent links. It is 
very easy to use with it's command line interface but we will add a web interface 
at the top of it to manage our downloads remotely from any web browser.

\begin{lstlisting}[language=bash,caption=Aria2 setup]
$ pacman -S aria2

# Download directory with access rights
$ mkdir /srv/downloads 
$ chmod o+w /srv/downloads

# Create configuration file with new example
$ nano /etc/aria2.conf
\end{lstlisting}

The configuration file can be set in any directory because you can specify to 
aria2 which file to use for it's configuration. In this example, this file is 
stored into \texttt{/etc} directory because it mostly contains configuration 
files but you can put it in your home directory.

\lstset{language=bash,caption=Aria2 configuration file}
\lstinputlisting{code/aria2.conf}

Now, you can use aria2 with it's command line interface but we will install the 
web interface before testing it and specify the configuration file directory.

\begin{lstlisting}[language=bash,caption=Install aria2 interface]
# Retrieve last release and put it in the usual directory
$ cd /usr/share/webapps/
$ git clone https://github.com/ziahamza/webui-aria2.git

# Create nginx domain or use this dirty way
$ ln -s /usr/share/webapps/webui-aria2/ /usr/share/nginx/html/aria2

# Setup and start a new service to run aria2 as a daemon
# See next example 
$ nano /usr/lib/systemd/system/aria2.service
$ systemctl start aria2

# You can try to download this image: 
# http://www.mumbly58.fr/wp-content/uploads/2013/01/archlinux-logo.png
$ ls /srv/downloads # must see image after download complete
\end{lstlisting}

Our web interface is directly connected to the aria2 program which have to be 
launched at every moment. Especially when you want to see your downloads through 
the web interface and even if they are stopped.
\\\\
Create a service which runs aria2 is the designated solution to complete this 
kind of tasks so here is an example.

\lstset{language=bash,caption=Run aria2 as a service}
\lstinputlisting{code/aria2.service}

\section{Torrent with rtorrent}

The previous software -- aria2 -- can handle torrents downloading like rtorrent 
but the second is dedicated and provides a very comfortable interface in the 
console. 
\\\\
I do not know what are the technical differences but if you want not 
use a web interface for torrents, rtorrent is the best solution in my opinion.

\subsection{rTorrent installation}
\begin{lstlisting}[language=bash,caption=rTorrent installation]
$ pacman -S rtorrent

# Torrents directory
$ mkdir /srv/torrent
$ chmod o+w /srv/torrent

# Session directory for rtorrent operations
$ mkdir /tmp/rtorrent
$ chmod o+w /tmp/rtorrent

# rTorrent configuration file, change the following options
# according to the two directories you created before
#
# directory = /srv/torrents
# session = /tmp/rtorrent
#
$ cp /usr/share/doc/rtorrent/rtorrent.rc ~/.rtorrent.rc
\end{lstlisting}

Considering there are no indications about how to use the rtorrent interface, I 
will the main commands to use for your torrents management.

\begin{table}[h]
\centering
	\begin{tabular}{lp{9cm}}
		\keys{\return} & Select torrent file to add (use \keys{\tab} to see current 
					 directory files)\\
					 
		\keys{\arrowkeyup} and \keys{\arrowkeydown} & Switch between seleced 
					 torrent\\
					 
		\keys{\arrowkeyright} & Visualize selected torrent details\\
	\\
		\keys{\ctrl + S} & Start selected torrent\\
		\keys{\ctrl + D} & Stop an active torrent and delete a stopped\\
		\keys{\ctrl + Q} & Quit rtorrent
	\end{tabular}	
\caption{Some rtorrent commands}
\end{table}

\subsection{Add torrents automatically}
% Add torrent auto

\subsection{ruTorrent web interface}
\subsubsection{Installation}
Despite a very good console interface you may prefer manage your torrents through 
a web interface which is fully multiplatform by nature. There are two steps to 
get a functionnal ruTorrent\footnote{The most known rtorrent web interface, some 
screenshot are available on the project github at \href{https://github.com/Novik/
ruTorrent}{github.com/Novik/ruTorrent}}: install it and then configure rtorrent 
and nginx to connect them with ruTorrent.

\begin{lstlisting}[language=bash,caption=ruTorrent installation and setup]
# Download ruTorrent in usual directory
$ cd /usr/share/webapps
$ git clone https://github.com/Novik/ruTorrent.git

# rtorrent configuration
# Add: scgi_port = 127.0.0.1:5000 in configuration file
$ nano ~/.rtorrent.rc

# nginx configuration
# Add this block in your domain for ruTorrent
location /RPC2 {
   scgi_port localhost:5000;
   include scgi_params;
}
\end{lstlisting}

At this point, you must be able to see the ruTorrent interface and visualize your 
torrents. If you still have no connexion with rtorrent check if it is launched and 
also the rutorrent configuration which must be good by default.

\begin{lstlisting}[language=bash,caption=Check ruTorrent configuration]
# ruTorrent configuration file
$ nano /usr/share/webapps/ruTorrent/conf/config.php

# Check that scgi setup is not set with sockets

# Must be set
$scgi_port = 5000;
$scgi_host = "127.0.0.1";

# Must be commented (socket)
// $scgi_port = 0;
// $scgi_host = "unix:///tmp/rpc.socket";
\end{lstlisting}

\subsubsection{Fix troubles}
When you access to ruTorrent web interface at the first you will see a lot of 
messages log section which warn you about some troubles in your configuration.

\begin{lstlisting}[language=bash,caption=Solve ruTorrent troubles]

Problem: 
Webserver user doesn't have read/write/execute access to the settings directory. ruTorrent settings cannot be saved. (/usr/share/webapps/ruTorrent/share/settings)

Solution:
$ chmod o+w  /usr/share/webapps/ruTorrent/share/torrents/

------------
Problems: 
Webserver user doesn't have read/write/execute access to the settings directory. ruTorrent settings cannot be saved. (/usr/share/webapps/ruTorrent/share/settings)
rTorrent user must have read/write/execute access to the settings directory. (/usr/share/webapps/ruTorrent/share/settings)
rTorrent user can't access 'stat' program. Some functionality will be unavailable.
scheduler: Plugin will not work. rTorrent user can't access external program (php).
ratio: Some functionality will be unavailable. rTorrent user can't access external program (php).
loginmgr: Some functionality will be unavailable. rTorrent user can't access external program (php).
history: Plugin will not work. rTorrent user can't access external program (php).
retrackers: Plugin will not work. rTorrent user can't access external program (php).
autotools: Plugin will not work. rTorrent user can't access external program (php).
datadir: Plugin will not work. rTorrent user can't access external program (php).
trafic: Plugin will not work. rTorrent user can't access external program (php).
unpack: Plugin will not work. rTorrent user can't access external program (php).
create: Plugin will not work. rTorrent user can't access external program (php).
rutracker_check: Plugin will not work. rTorrent user can't access external program (php).
rss: Plugin will not work. rTorrent user can't access external program (php).
screenshots: Plugin will not work. rTorrent user can't access external program (ffmpeg).
_task: Some functionality will be unavailable. rTorrent user can't access external program (pgrep).
rss: Some functionality will be unavailable. rTorrent user can't access external program (curl).

Solution:
$ chmod o+w  /usr/share/webapps/ruTorrent/share/settings/

-----------
Problem:
Unpack plugin: rTorrent user can't access 'unrar' program.
Cause of solved: unpack: Plugin will not work. rTorrent user can't access external program (php).

Solution:
$ pacman -S unrar

-----------
[06.01.2015 11:14:45] Webserver user can't access 'stat' program. Some functionality will be unavailable.
[06.01.2015 11:14:45] rTorrent user must have read/execute access to the tmp directory. ruTorrent will not work. (/tmp/)
[06.01.2015 11:14:45] rss: Some functionality will be unavailable. Webserver user can't access external program (curl).
[06.01.2015 11:14:45] rutracker_check: Plugin will not work. Webserver user must have execute access to the rtorrent session directory (/tmp/rtorrent/).
[06.01.2015 11:14:45] mediainfo: Plugin will not work. rTorrent user can't access external program (mediainfo).

\end{lstlisting}

\section{Owncloud}
\begin{lstlisting}[language=bash,caption=Owncloud setup]
$ cd /usr/share/webapps
$ wget https://download.owncloud.org/community/owncloud-latest.tar.bz2
$ tar -xvf owncloud-latest.tar.bz2  # long time
$ rm owncloud-latest.tar.bz2 

# http://doc.owncloud.org/server/6.0/admin_manual/installation/installation_source.html#set-the-directory-permissions
$ chown root:root /usr/share/webapps/owncloud
$ chown -R http:http /usr/share/webapps/owncloud/*

$ nano /etc/nginx/site-available/owncloud
$ ln -s /etc/nginx/site-available/owncloud /etc/nginx/site-enabled/

--------
Problem: 
PHP module zip not installed.
Please ask your server administrator to install the module.

Solution:
# Uncomment extension=zip.so
$ nano /etc/php/php.ini 
$ systemctl reload php-fpm

-------
Problem:
PHP module GD not installed.
Please ask your server administrator to install the module.

Solution:
$ pacman -S php-gd
# Uncomment extension=gd.so
$ nano /etc/php/php.ini 
$ systemctl reload php-fpm

------
$ mkdir /srv/owncloud
$ chmod o+w /srv/owncloud

# db pass changed
# add /srv/owncloud to open_basedir
$ nano /etc/php/php.ini
$ systemctl reload php-fpm
\end{lstlisting}
\end{document}
