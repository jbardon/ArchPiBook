\chapter{Introduction}
	\section{Are you interested?}
This book is written by a no-specialist of ArchLinux with basic knowledge of 
linux system so I will try to made it as simple as possible for people 
who have no idea about what is console. Indeed, all commands will be 
explained for a better comprehension and an index will be available for you.

No matter if you are an expert or a novice, you will be able to find 
how to install stuff on your your Pi plus tips which includes all the problems 
I encounter during my first installation.

	\section{What is a Raspberry}
If you succeed to find this book I guess you allready know but some people 
buy one with OpenELEC\footnote{Tiny linux system based on XBMC media center. 
More details on \href{http://openelec.tv}{openelec.tv}} pre-installed so here 
is a little explaination.

The Raspberry Pi is a credit-size computer with low performance if you
compare with a common PC. Nevertheless, it means its power consumption is
very low (1W for B+ version\footnote{Most robust version of RPi with 512MB 
of RAM and 4 usb ports}) so it is not a problem to let it on forever. 

Now if you install a good linux distribution on it you can turn this old computer
into a cheap server on which you will have the control. You can use it 
just at home for file sharing, media player or others but it is also possible
to host a website which will be available on the internet\footnote{An example
of website hosted by a Rpi on \href{http://raspberrypi.goddess-gate.com}
{raspberrypi.goddess-gate.com}}.

	\section{Why ArchLinux and not \href{http://www.raspbian.org}{Raspbian}}